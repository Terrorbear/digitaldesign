\documentclass[12pt]{report}

\usepackage[utf8]{inputenc}
\usepackage{graphicx}
\graphicspath{ {images/} }

\usepackage[a4paper,width=150mm,top=25mm,bottom=25mm]{geometry}

\usepackage{fancyhdr}
\pagestyle{fancy}

% For citations
\usepackage[backend=bibtex]{biblatex}
\addbibresource{Final_Paper.bib}
\usepackage{hyperref}
\hypersetup{
  linkcolor=blue,
  citecolor=blue,
  urlcolor=blue
}

\usepackage{xcolor}
\definecolor{codebgnd}{rgb}{0.9, 0.9, 0.9}
\usepackage{listings}
\lstset{
  basicstyle=\footnotesize\ttfamily,
  breaklines=true,
  backgroundcolor=\color{codebgnd}
}

\title{
  {Ethernet on the Zynq ZC706 \vspace{0.2in}}\\
  {\large 18-545 Advanced Digital Design \vspace{0.2in}}\\
  {\includegraphics[width=2in]{cmu_seal.png}}
}

\author{Terence An, Eddie Nolan, Dale Zhang}
\date{December 12, 2015}

\begin{document}
\maketitle

\chapter{Introduction}
This paper is a guide to start building ethernet on the Zynq ZC706 board. It was originally written as the final project for a 18-545 project which didn't complete because they struggled to build an ethernet adapter in programmable logic and have it properly communicate with the Processing System. The intention of this paper is to aid future groups in completing an ethernet adapter, as well as providing the necessary background and deterring groups from fruitless avenues. This guide will expect a very minimal understanding of Vivado because most students in 18-545 have had very limited exposure to Vivado. We will attempt to provide the pertinent references as needed.

That being said, going through Lab 2 in \textit{Vivado Design Suite Tutorial} \cite{vivado_tut} will probably be the fastest way to understand the work flow. Also, chapter 2 and chapter 4 of \textit{UltraFast Design Methodology Guide for the Vivado Design Suite} \cite{ultrafast} will be superbly helpful in learning to use Vivado, especially for using IP in Vivado. Finally, if you still want more details on using IP, you can refer to the Vivado guide on \textit{Designing with IP} \cite{IP} and \textit{Designing IP Subsystems Using IP Integrator} \cite{IP_subsystems}.


\chapter{Ethernet Background}

\chapter{Basic Approaches}

\chapter{Ethernet PL Guide}

\chapter{PetaLinux Networking}
\subsection{Introduction}
This section will focus on the attempts we made to implement the software-firmware stack that Xilinx provides through their application notice XAPP1082 and through a guide on the Xilinx wiki. This involves creating a bootable SD card containing a copy of PetaLinux, Xilinx's embedded Linux distribution, that is patched to support communicating over the SFP Ethernet port, and flashing the FPGA fabric with a corresponding bitstream that implements the hardware Ethernet support. Although our eventual goal was to modify the kernel and firmware to support Ethernet traffic analysis, we ran into bugs that we were unable to fully diagnose and solve.
\subsection{Relevant Hyperlinks and Documentation}
\begin{itemize}
  \item \href{http://www.xilinx.com/support/documentation/application_notes/xapp1082-zynq-eth.pdf}{XAPP1082 PDF}
  \item Xilinx Wiki Guides:
    \begin{itemize}
    \item \href{http://www.wiki.xilinx.com/Zynq+PL+Ethernet}{Zynq Ethernet Guide}
    \item \href{http://www.wiki.xilinx.com/Prepare+Boot+Medium}{Zynq Boot Setup Guide}
    \end{itemize}
  \item \href{http://www.xilinx.com/support/documentation/boards_and_kits/zc706/ug954-zc706-eval-board-xc7z045-ap-soc.pdf}{Zynq ZC706 User Guide}
  \item \href{http://www.xilinx.com/support/documentation/sw_manuals/petalinux2014_4/ug1144-petalinux-tools-reference-guide.pdf}{PetaLinux Tools Reference Guide}
  \item \href{http://www.syfer.com.au/assets/s502-00000-a.pdf}{Installation Guide for PetaLinux 2014.4 (Not by Xilinx)}
  \item \href{http://www.silabs.com/products/mcu/Pages/USBtoUARTBridgeVCPDrivers.aspx}{Drivers for Zynq ZC706 UART}
\end{itemize}
\subsection{Supplies}
\begin{itemize}
  \item Hardware: 
    \begin{itemize}
    \item It's necessary to be able to connect to a wired Ethernet connection and write SD cards with the computer you are using for development.
    \item If you want to switch between different copies of PetaLinux, it's convenient to have multiple SD cards, so you don't have to reflash them repeatedly. The copy of PetaLinux we used was only ~200MB so large storage capacity is unneccessary.
    \item It's helpful to have an Ethernet switch or router so you can test the device without needing access to the full Internet.
    \end{itemize}
  \item Software:
    \begin{itemize}
    \item I installed the PetaLinux tools on two Ubuntu LTS virtual machines. You should have virtualization software and enough disk space available on your computer for 2 virtual machines.
    \item We also were able to use lab computers that were configured to use the version of Vivado that corresponded to the version of XAPP1082 we used.
    \item The specific versions that our project used were XAPP1082 version 3, PetaLinux and Vivado version 2014.4, Xubuntu 14.04 LTS for installing the PetaLinux tools. Our lab machines ran RHEL.
    \end{itemize}
\end{itemize}

\section{Using PetaLinux}
\subsection{Creating Boot Media}
The Zynq Boot Setup Guide linked above demonstrates the way to format bootable SD cards. You need to create 2 partitions, \texttt{boot} and \texttt{root}, and copy 2 files, \texttt{BOOT.bin} and \texttt{image.ub}, onto the 'boot' partition, but the partition table needs to be configured in a specific way using fdisk commands. The relevant part of the guide is the section under ``SD Boot''; ignore the sections that describe JTAG and QSPI booting. You can find relevant precompiled \texttt{BOOT.bin} and \texttt{image.ub} files in the XAPP1082 software release: \texttt{\$XAPP\_HOME/ready\_to\_test/pl\_eth\_noCso}.

\subsection{Building XAPP1082's PetaLinux Configuration}
The version of the PetaLinux Tools that we used wouldn't install properly on a 64-bit virtual machine because of issues with 32-bit library support that we were unable to resolve. However, the Xilinx SDK, which is required to create the boot image after the kernel has been compiled, only supports 64-bit versions of Linux. We were able to get around this by creating a 32-bit VM for creating the project and building the kernel, and a separate 64-bit VM for creating the boot image.

\subsubsection{Setting up the 32-bit VM}
For Ubuntu, the PetaLinux Tools Reference Guide required the following packages to be installed:
\newline \texttt{tofrodos iproute gawk gcc git-core make net-tools libncurses5-dev tftpd zlib1g-dev flex bison}

The PetaLinux Tools are distributed as a self-extracting archive/installer. The Xilinx Wiki Zynq Ethernet Guide contains a download link. Both the PetaLinux Tools and the XAPP1082 software release should be extracted. The provided shell scripts help to automate steps that are described in the Wiki guide. The directories on the first lines should be modified to reflect the user's configuration.


\subsubsection{Setting up the 64-bit VM}
To set up the 64-bit VM, you need to install all the same packages as the 32 bit VM needs. In addition to extracting the PetaLinux installer, you also need to install the Xilinx SDK that corresponds to your version of PetaLinux. After you have finished configuring and building the project on the 32-bit VM, you can copy the \texttt{xapp1082\_pl\_eth} project directory to the 64-bit VM, then use the provided shell script to create a bootable image (more detail is available in the wiki guide). The required files for creating a bootable SD card will be in the project directory in the subdirectory \texttt{images/linux}.

\subsubsection{Note on PetaLinux Setup}
In an attempt to diagnose the main problem that we encountered, we needed to introduce a change to the Xilinx ethernet driver so that it would print debug information. We did this by text-editing the patch file that adds driver support to the kernel, although there are many other ways of modifying the PetaLinux configuration and software that are described in more detail in the documentation. If there is no need to customize PetaLinux in any way, precompiled images are provided in the XAPP1082 software release, and the PetaLinux Tools don't need to be used.

\subsection{Connecting to the UART of the Zynq ZC706}
The driver download page for the ZC706 can be found in the link section. After the drivers are installed, remote terminal software can be used to communicate with the ZC706. We found it easiest to use \texttt{screen}. If the UART shows up as \texttt{/dev/ttyUSB0}, the ZC706 can be connected to by running \texttt{screen /dev/ttyUSB0 115200}. Permissions issues can often prevent reading from device files so it's important to ensure that permissions are set properly.

\subsection{PetaLinux Networking Setup}
\subsubsection{SFP Ethernet Port}
As described in the Xilinx Wiki guide, to activate the interface for the SFP Ethernet port, you should run
\newline \texttt{insmod /lib/modules/3.17.0-xilinx-<kernel\_build\_version>/kernel/}
\newline \texttt{drivers/net/ethernet/xilinx/xilinx\_axienet\_main.ko}
\newline This loads the kernel module that enables the SFP Ethernet interface.
\newline To activate the interface, run
\newline \texttt{ifconfig eth1 up x.x.x.x}
\newline where \texttt{x.x.x.x} is the IP address assigned to the SFP interface.
\newline You should see the following two messages:
\newline \texttt{eth1: XAXIEthernet: PHY Link carrier lost.}
\newline \texttt{eth1: XAXIEthernet: PHY Link carrier restored.}
\newline If the second message does not appear, that means that Petalinux is unable to set the link up. This is the problem that our team encountered that we were unable to move past.

\subsubsection{Configuration}
Unfortunately, the tools that are included by default in the provided PetaLinux configuration are limited; in particular, there is no DHCP support. In order to set up an interface, you will need to know you're device's:
\begin{itemize}
\item Assigned IP address
\item Broadcast address
\item Network mask
\item Default gateway
\end{itemize}
The default gateway can be set by running:
\newline \texttt{route add default gateway x.x.x.x}
\newline You can set the IP address and activate the interface by running:
\newline \texttt{ifconfig eth0 up x.x.x.x}
\newline Use ifconfig to add the netmask and broadcast address:
\newline \texttt{ifconfig eth0 broadcast x.x.x.x}
\newline \texttt{ifconfig eth0 netmask x.x.x.x}
\newline If your configuration works, you should be able to connect to the internet. 

\subsection{Bugs and Roadblocks}
The bug that we were unable to overcome in our project was that Petalinux was unable to successfully set the link up for the SFP port interface. The Xilinx Wiki advised us to make sure we saw the message
\newline \texttt{eth1: XAxiEthernet: PHY Link carrier restored.}
\newline before using the SFP interface. We never encountered this message. Here is the part of the driver code that triggers this message:
\begin{lstlisting}[language=C]
if (phy_carrier) {
  printk(KERN_INFO
         "%s: XAxiEthernet: PHY Link carrier restored.\n",
         dev->name);
  netif_carrier_on(dev);
  set_mac_speed(lp);
}
\end{lstlisting}
The variable \texttt{phy\_carrier} is set to the value of \texttt{linkup} in the \texttt{get\_phy\_status} function:
\begin{lstlisting}[language=C]
static int get_phy_status(struct net_device *dev, DUPLEX * duplex, int *linkup)
{
	struct axienet_local *lp = (struct axienet_local *) netdev_priv(dev);
	u32 reg1, reg2;

	//reg1 = axienet_mdio_read_local(lp, lp->gmii_addr, MII_BMCR);
	reg1 = axienet_mdio_read(lp->mii_bus, lp->gmii_addr, MII_BMCR);
	*duplex = FULL_DUPLEX;

	//reg2 = axienet_mdio_read_local(lp, lp->gmii_addr, MII_BMSR);
	reg2 = axienet_mdio_read(lp->mii_bus, lp->gmii_addr, MII_BMSR);
	*linkup = (reg2 & BMSR_LSTATUS) != 0;

	//printk(KERN_ERR "get_phy_status: BMCR=0x%x, BMSR=0x%x, duplex=%d, linkup=%d\n", reg1, reg2, *duplex, *linkup);
	return 0;
}
\end{lstlisting}
Reading the driver code showed that the status and control values that the SFP device was providing to the MII interface didn't indicate that the Ethernet link was up. We modified the driver code to uncomment the call to \texttt{printk}, which showed us that the MII control register value was always 0x1140, while the status register was always 0x01c8. The values are bitmasks that we interpreted the following way:
\begin{verbatim}
MII Basic Mode Status Register
Bit 0: Extended MII registers available
Bit 1: Jabber detected (sticky)
Bit 2: Link is up (sticky)
Bit 3: Capable of auto-negotiation
Bit 4: Remote fault detected
Bit 5: Auto-negotiation complete
Bits 6-A: unused
Bit B: Capable of 10mbps half-duplex
Bit C: Capable of 10mbps full-duplex
Bit D: Capable of 100mbps half-duplex
Bit E: Capable of 100mbps full-duplex
Bit F: Capable of 100mbps 4k packets

Value we get from ZC706: 0x01c8

 F E D C B A 9 8 7 6 5 4 3 2 1 0
|0|0|0|0|0|0|0|1|1|1|0|0|1|0|0|0|

MII Basic Mode Control Register
Bits 0-6: unused
Bit 7: Collision test
Bit 8: Full duplex
Bit 9: Autonegotiation restart
Bit A: Disconnect PHY from MII
Bit B: Power down PHY
Bit C: Enable autonegotiation
Bit D: Select 100mbps
Bit E: TXD loopback bits
Bit F: Reset

Value we get from ZC706: 0x1140

 F E D C B A 9 8 7 6 5 4 3 2 1 0
|0|0|0|1|0|0|0|1|0|1|0|0|0|0|0|0|
\end{verbatim}

Some of these reported settings and status bits make sense, but some do not; no status bit is set that says that the PHY is able of any particular Ethernet link speed, but at the same time both the status and control registers say that the device is configured for autonegotiation.

This was as far as we were able to get with debugging this issue. Something we would have wanted to test was whether the SFP port acted differently with a different module. The XAPP1082 PDF specified that it was tested with a ``HP 378928-B21 Cisco Gigabit Ethernet RJ45 SFP Module,'' whereas our SFP module was a different brand and model.

\subsection{Other Notes}
\begin{itemize}
\item Make sure jumper J17 is set (see XAPP1082 PDF)
\end{itemize}
\subsection{Code Listing}
\subsubsection{Shell Scripts for the 32-bit VM}
\texttt{1\_create\_project\_config\_kernel.sh}
\lstinputlisting{shellscripts/1_create_project_config_kernel.sh}

\texttt{2\_apply\_patch\_build\_kernel.sh}
\lstinputlisting{shellscripts/2_apply_patch_build_kernel.sh}

\subsubsection{Shell Script for the 64-bit VM}
\texttt{3\_create\_boot\_image.sh}
\lstinputlisting{shellscripts/3_create_Boot_image.sh}
\chapter{Alternate Approaches}

\chapter{More on PL}

\chapter{Miscellaneous}
\section{Personal Statements}
\subsection{Dale Zhang}
For me, this class was challenging for a multitude of reasons. First off, I hadn't touched Verilog since I took 18-240 a few years ago, and I also wasn't very comfortable with HDL. In addition, in the context of our project, I had very little knowledge on ethernet and networking, so I had to learn a lot about how our project worked as I went along.

Over the course of the semester, there are definitely a few things I wish I had done differently. Since Terence and Eddie were both far more knowledgeable about Ethernet and networking, I often took a backseat to them when the group was making decisions. However, at some points in the semester, instead of asking for their help understanding some of the concepts driving our design, I would try to do it myself, without very much success. This led to me spending far more time on some tasks than I should've. This definitely limited my effectiveness as a team member.

Another mistake we made as a team was underestimating how much work actually needed to go into this project. Towards the beginning of the semester, we didn't put in much lab time outside of class periods and mandatory lab time. It first really caught up to us around mid semester with the first status meeting, where we saw how far behind we were, and how much more time we would need to commit for the rest of the semester.

Some advice I'd have for anyone planning to pursue an FPGA Ethernet project in the future is not to spend too much time trying to do research, and to start actually working on the board as soon as possible. In addition, ethernet on FPGA is not very well documented, and much of the documentation available is incorrect or incomplete.

For the class in general, it's definitely better to spend the long hours working on your project earlier in the semester, before your other classes have started to pick up. In addition, at the beginning of the semester, try and pick a project that you can be passionate about and that you would really like to see succeed. At times, I felt very unmotivated to go in an work on the project simply because I wasn't particularly excited about our final product.

To all future students reading this, good luck with the class and have fun!

\printbibliography
\appendix
\chapter{Appendix}


\end{document}
