\documentclass[12pt]{report}

\usepackage[utf8]{inputenc}
\usepackage{graphicx}
\graphicspath{ {images/} }

\usepackage[a4paper,width=150mm,top=25mm,bottom=25mm]{geometry}

\usepackage{fancyhdr}
\pagestyle{fancy}

\title{
  {Ethernet on the Zynq ZC706 \vspace{0.2in}}\\
  {\large 18-545 Advanced Digital Design \vspace{0.2in}}\\
  {\includegraphics[width=2in]{cmu_seal.png}}
}

\author{Terence An, Eddie Nolan, Dale Zhang}
\date{December 12, 2015}

\begin{document}
\maketitle

\chapter{Introduction}

\chapter{Basic Approaches}

\chapter{Ethernet PL Guide}

\chapter{PetaLinux Networking}

\chapter{Alternate Approaches}

\chapter{More on PL}

\chapter{Miscellaneous}
\section{Personal Statements}
\subsection{Dale Zhang}
For me, this class was challenging for a multitude of reasons. First off, I hadn't touched Verilog since I took 18-240 a few years ago, and I also wasn't very comfortable with HDL. In addition, in the context of our project, I had very little knowledge on ethernet and networking, so I had to learn a lot about how our project worked as I went along.

Over the course of the semester, there are definitely a few things I wish I had done differently. Since Terence and Eddie were both far more knowledgeable about Ethernet and networking, I often took a backseat to them when the group was making decisions. However, at some points in the semester, instead of asking for their help understanding some of the concepts driving our design, I would try to do it myself, without very much success. This led to me spending far more time on some tasks than I should've. This definitely limited my effectiveness as a team member.

Another mistake we made as a team was underestimating how much work actually needed to go into this project. Towards the beginning of the semester, we didn't put in much lab time outside of class periods and mandatory lab time. It first really caught up to us around mid semester with the first status meeting, where we saw how far behind we were, and how much more time we would need to commit for the rest of the semester.

Some advice I'd have for anyone planning to pursue an FPGA Ethernet project in the future is not to spend too much time trying to do research, and to start actually working on the board as soon as possible. In addition, ethernet on FPGA is not very well documented, and much of the documentation available is incorrect or incomplete.

For the class in general, it's definitely better to spend the long hours working on your project earlier in the semester, before your other classes have started to pick up. In addition, at the beginning of the semester, try and pick a project that you can be passionate about and that you would really like to see succeed. At times, I felt very unmotivated to go in an work on the project simply because I wasn't particularly excited about our final product.

To all future students reading this, good luck with the class and have fun!
\appendix
\chapter{Appendix}

\end{document}
